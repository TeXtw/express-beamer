\section[Problem]{HyperLMNtal 定式化における課題}
\begin{frame}{The difficulty in defining fusion}

  \fbox{$(a(!X), (!X \bowtie\ !Y, b(!X, !Y)))$}
  が
  \begin{itemize}
  \item[$\equiv$] 
    \fbox{$(a(!X), b(!Y, !Y))$}
  \end{itemize}
  になるとまずい
  
  Therefore, we cannot just define the structural congruence rule as  
  \[(!X \bowtie\ !Y, P) \equiv P[!Y/!X]
  \mbox{\hspace{2em} if \(!X\) occurs free in \(P\)}\]
  
  We must ensure that there is no \mfbox{!X} occurs outside of the \mfbox{P}\\
  That is, \mfbox{!X} should be a \emph{local link} of \mfbox{(!X \bowtie\ !Y, P)}
\end{frame}

\begin{frame}{Hyper でない LMNtal ではなぜ大丈夫なのか?}

  再掲:(E9)
  $(X = Y, P) \equiv P[Y/X]$
  \hspace{1em}
  if $P$ is an atom and \underline{$X$ occurs free in $P$}

  \vspace{2em}
  
  In (not-hyper) LMNtal, a link appeared twice is a local link
  \begin{itemize}
    \thusitem
    the condition of (E9), \underline{``\textit{if \(X\) occurs free in \(P\)}''}\\
    requires the \mfbox{X} to be a local link in \mfbox{(X = Y, P)}
    \item[\(\because\)]
      \mfbox{X} occurs in both \mfbox{X = Y} and \mfbox{P}.
  \end{itemize}

\end{frame}

\begin{frame}{ダメな回避方法}

  This problem would not happen if we always ``leave'' the \mfbox{!X \bowtie\ !Y}

  For example, as 
  \[(!X \bowtie\ !Y, P) \equiv\ (!X \bowtie\ !Y, P[!Y/!X]) \]
  
  でも結局,\mfbox{!X} の局所性がわからないと\\
  いつまで経っても \mfbox{!X \bowtie\ !Y} を消して良いのかわからない.

  If we cannot diminish the \mfbox{!X \bowtie\ !Y},\\
  then we cannot create it neither (by applying the rule reversely).
  
  Therefore, the existence of the \mfbox{!X \bowtie\ !Y} would let the process
  different from that does not have it,
  which is certainly not the desired behavior based on the current implementation.
  
\end{frame}

\begin{frame}{結論}
  Hyperlink にも \Emph{局所性を定義する必要がある}

  \begin{itemize}
  \item
    ただしハイパーリンクは出現回数を数えるだけでは\\
    局所性の判別ができない
    \thusitem
    パイ計算\footcite{pi}にならって \mfbox{\nu} を入れて\\
    明示的に局所性を導入した
  \end{itemize}
\end{frame}

