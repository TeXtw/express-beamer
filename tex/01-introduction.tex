\newcommand{\code}[1]{\fbox{\texttt{#1}}}

\begin{frame}{発表概要}
  \begin{description}
  \item[目的]
    HyperLMNtal\footcite{hyperlmntal} に \Emph{形式的な定義} を与えたい
  \item[背景]
    Hyperlink は \Emph{実装済みだが形式的定義はなかった}
  \item[貢献]
    HyperLMNtal に対して形式的定義を与え,\\
    重要な性質については証明を行なった
  \end{description}
\end{frame}


\section[Background]{背景:LMNtal・HyperLMNtal}

\begin{frame}{グラフ書き換え言語 LMNtal\footcite{logiclmntal}}
  LMNtal は\dots
  \begin{description}
  \item[特徴]
    1対1接続の \emph{リンク \small (辺)}によってグラフを構成する\\
    \emph{アトム \small (ラベル付き頂点)} の多重集合を\\
    \emph{ルール} のパターンマッチと書き換えによって計算を進める
  \item[利点]
    Null,Dangling pointer を発生させずにグラフを扱える
  \end{description}

  ただし,\emph{1対1接続のリンクしかないと不便} なこともある
  \begin{itemize}
    \item 動的に被参照数が変化する共有データの実装など
  \end{itemize}

  そこで,任意数の端点を持てる \Emph{Hyperlink} が実装されている
\end{frame}


\begin{frame}{LMNtal の拡張|HyperLMNtal\footcite{hyperlmntal}}

  Hyperlink は \code{\$x} か \code{!X} のように記述する

  \lstinputlisting[language=lmntal]{fig/hyperlinks.lmn}

\end{frame}



\begin{frame}{HyperLMNtal の問題点|形式的定義がない}
  \begin{itemize}
  \item
    先述のコードの実行結果(と過程)を保証するものは\\
    実装とそのアルゴリズムだけ
    \begin{itemize}
    \item
      コードの挙動を正しく理解するために\\
      コンパイラ(と実行時処理系)まで理解する必要がある
    \end{itemize}
  \item
    今後の発展のためには形式的議論に耐え得るような定義が不可欠
    \begin{itemize}
    \item
      拡張や制約の追加などを行う際に\\
      そもそも元々の定義がないとまともな議論を期待できない
    \end{itemize}
    
  \end{itemize}  
\end{frame}
